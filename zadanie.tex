\documentclass[12pt]{article}
\usepackage[MeX]{polski}
\usepackage[utf8]{inputenc}
\usepackage{graphicx}
\usepackage{amsmath} %pakiet matematyczny
\usepackage{amssymb} %pakiet dodatkowych symboli
\begin{document}
Lorem ipsum dolor sit amet, consectetur adipiscing elit. Curabitur suscipit iaculis turpis pharetra auctor. Aenean ac ante \textbf{sit} 
amet elit rhoncus aliquam. Mauris a erat elementum neque volutpat fermentum sit amet interdum diam. Duis quis augue \underline{neque} .
Sed porttitor justo sed velit sollicitudin sagittis. Mauris suscipit malesuada diam. Aenean congue arcu at eros, \textbf{\textit{tristique}}
\emph{quis} imperdiet felis faucibus. Integer feugiat pulvinar rutrum. Duis eget dui sapien. Nulla rutrum leo vel ante ullamcorper 
tristique. Phasellus mauris lacus, condimentum quis luctus sollicitudin, facilisis sed risus. Aenean sed leo ac augue 
\textbf{ursus} ornare et in eros. Praesent quis nisl nisl. Integer laoreet aliquet est, in luctus purus congue id. Nullam 
vitae luctus arcu. Nullam rutrum sapien sit amet vestibulum tristique.
$$
\begin{array}{|l|cr}
left1 & center 1 & right 1 \\ hline
d & e & f \\
\end{array}
$$
$$
\begin{array}{lll}
z &=& a \\
	&=& a\\
f(x,y,z) &=& x+y+z \\
\end{array}
$$
$$
\chi(x)=
\begin{array}{ccc}
x-a & -b & -c \\
-d & x-e & -f \\
-g & -h & x-i \\
\end{array}
$$
$$
\chi(x)=\left|
\begin{array}{ccc}
x-a & -b & -c \\
-d & x-e & -f \\
-g & -h & x-i \\
\end{array} \right|
$$
$$
\left[
\begin{array}{c|c|c|c}
A & Ab & \cdots & A^{n-1} b\\
\end{array}
\right]
$$
$$
\left[
\begin{array}{cccc|c}
a_{11} & a_{12} & \cdots & a_{1n} & b_1 \\
a_{21} & a_{22} & \cdots & a_{2n} & b_2 \\
\vdots & \ddots & \cdots & vdots &  \\
a_{n1} & a_{n2} & \cdots & a_{nn} & b_n \\
\end{array}
\right]
$$
$$
\left(
\begin{array}{ccccc}
1 & 2 & 3 & 4 &5 \\
-10 & -20 & -30 & -40 & -50 \\
\\
\ldots & 0 & \ldots & 0  & \ldots \\
\end{array}
\right)
$$
$\lim_{n \to \infty}$
$\lim\limits _{n \to \infty}$
$$\lim _{n \to \infty}
\left(
1-\frac{1}{a^n}
\right)^n=e
$$
$$
\int _a^b \sin(x) \, dx
$$
$$
\iint\displaylimits _D(x^2+y-4) \, dx \, dy
$$
\textbf{Wyrażenie \#1}
\\
4.Korzystając z granic właściwej i niewłaściwych ciągu uzasadnić podane równości:
\\
$$
a)
\lim \limits_{n \to \infty} \frac{n+1}{n}=1
 \hspace{2mm} b)
\lim \limits_{n \to \infty} \frac{(-1)^n}{n}=0
 \hspace{2mm} c)
\lim \limits_{n \to \infty} \frac{n+1}{n}=2
 \hspace{2mm} d)
\lim \limits_{n \to \infty} \frac{n+1}{n}=\infty
 \hspace{2mm} e)
\lim \limits_{n \to \infty} \frac{n+1}{n}=-\infty
$$
\end{document}
